% Options for packages loaded elsewhere
\PassOptionsToPackage{unicode}{hyperref}
\PassOptionsToPackage{hyphens}{url}
%
\documentclass[
]{article}
\usepackage{lmodern}
\usepackage{amssymb,amsmath}
\usepackage{ifxetex,ifluatex}
\ifnum 0\ifxetex 1\fi\ifluatex 1\fi=0 % if pdftex
  \usepackage[T1]{fontenc}
  \usepackage[utf8]{inputenc}
  \usepackage{textcomp} % provide euro and other symbols
\else % if luatex or xetex
  \usepackage{unicode-math}
  \defaultfontfeatures{Scale=MatchLowercase}
  \defaultfontfeatures[\rmfamily]{Ligatures=TeX,Scale=1}
\fi
% Use upquote if available, for straight quotes in verbatim environments
\IfFileExists{upquote.sty}{\usepackage{upquote}}{}
\IfFileExists{microtype.sty}{% use microtype if available
  \usepackage[]{microtype}
  \UseMicrotypeSet[protrusion]{basicmath} % disable protrusion for tt fonts
}{}
\makeatletter
\@ifundefined{KOMAClassName}{% if non-KOMA class
  \IfFileExists{parskip.sty}{%
    \usepackage{parskip}
  }{% else
    \setlength{\parindent}{0pt}
    \setlength{\parskip}{6pt plus 2pt minus 1pt}}
}{% if KOMA class
  \KOMAoptions{parskip=half}}
\makeatother
\usepackage{xcolor}
\IfFileExists{xurl.sty}{\usepackage{xurl}}{} % add URL line breaks if available
\IfFileExists{bookmark.sty}{\usepackage{bookmark}}{\usepackage{hyperref}}
\hypersetup{
  pdftitle={Poročilo projekta - Voronoijevi diagrami},
  pdfauthor={Anže Mramor in Peter Tiselj},
  hidelinks,
  pdfcreator={LaTeX via pandoc}}
\urlstyle{same} % disable monospaced font for URLs
\usepackage[margin=1in]{geometry}
\usepackage{graphicx,grffile}
\makeatletter
\def\maxwidth{\ifdim\Gin@nat@width>\linewidth\linewidth\else\Gin@nat@width\fi}
\def\maxheight{\ifdim\Gin@nat@height>\textheight\textheight\else\Gin@nat@height\fi}
\makeatother
% Scale images if necessary, so that they will not overflow the page
% margins by default, and it is still possible to overwrite the defaults
% using explicit options in \includegraphics[width, height, ...]{}
\setkeys{Gin}{width=\maxwidth,height=\maxheight,keepaspectratio}
% Set default figure placement to htbp
\makeatletter
\def\fps@figure{htbp}
\makeatother
\setlength{\emergencystretch}{3em} % prevent overfull lines
\providecommand{\tightlist}{%
  \setlength{\itemsep}{0pt}\setlength{\parskip}{0pt}}
\setcounter{secnumdepth}{-\maxdimen} % remove section numbering

\title{Poročilo projekta - Voronoijevi diagrami}
\author{Anže Mramor in Peter Tiselj}
\date{4.12.2020}

\begin{document}
\maketitle

\hypertarget{poroux10dilo-projekta---voronoijevi-diagrami}{%
\section{Poročilo projekta - Voronoijevi
diagrami}\label{poroux10dilo-projekta---voronoijevi-diagrami}}

\hypertarget{uvod---definicija-vornoijevih-diagramov}{%
\subsection{Uvod - Definicija Vornoijevih
diagramov}\label{uvod---definicija-vornoijevih-diagramov}}

Voronoijeve diagrame definiramo na ravnini ali na splošnem metričnem
prostoru. V danem grafu \(G\) in podmnožici vozlišč \(U\)
(\(U \subseteq G\)) razdelimo vozlišča grafa \(G\), glede na to, katero
vozlišče iz \(U\) jim je najbližje. Bolj natančno: za
\(\forall u \in U\), definiramo Voronoijevo celico, \((u, U)\), kot
množico vseh vozlišč iz \(G\), ki so bližje \(u\) kot kateremu koli
drugemu vozlišu \(U \setminus \{u\}\).\textbackslash{} Oziroma s
formalno definicijo:\textbackslash{} Naj bo \(X\) metrični prostor z
razdaljo \(d\), ki je definirana kot razdalja med točko \(x \in X\) in
podmnožico \(A\) : \(d(x, A) = inf \{d(x, A) | a \in A \}\) . Naj bo še
\(K\) množica indeksov in \((P_k)_{k \in K}\) terica (tuple) nepraznih
podmnožic v \(X\). Potem definiramo Voronoijevo celico kot \(R_k\) za
\(k \in K\) kot
\[ R_k = \{ x \in X | d(x,P_{k}) \leq d(x,P_{j}) ; \forall j \neq k \} \]
Voronoijev diagram je enostavno terica Voronoijevih celic
\((R_k)_{k \in K}\)

\hypertarget{opis-problema}{%
\subsection{Opis problema}\label{opis-problema}}

Pogledala sva si enostavna omrežja: 1. dvo- in trodeminzionalne mreže
različnih velikosti - $1 \times n, 2 \times n, n \times n, \dots $ 2.
različne tipe grafov kot so binarna drevesa in grafi s cikli

Za vsako omrežje sva si izbrala več smiselnih vrednosti \(k\), ki so
nama predstavljali število naključno izbranih vozlišč, za katere sva
izračunala Voronoijeve celice. Nato sva statistično obdelovala velikosti
dobljenih Voronoijevih celic. Pogledala sva povprečno število vozlišč v
minimalni, maksimalni in povprečni Voronoijevi celici za vsako omrežje
in vse izbrane \(k\), poskušala iz grafov z manjšim številom vozlišč
napovedati kakšno bo število vozlišč v celicah pri grafih z večjim
številom vozlišč, ter poskušala ugotoviti kakšna je porazdelitev
velikosti celic glede na izbrani \(k\) (Peter dopolni primerno kar si
delu pls).\\
Za vsak izbrani \(k\) sva program pognala večkrat, saj sva tako lahko
opazovala kako se pri različnem izboru začetnih vozlišč, spreminjajo
zgornje vrednosti. Rezultate sva shranjevala v \emph{.tsv} datoteke in
jih nato obdelala v programu \emph{r}. Ker so tabele za najin problem
zelo velike, bova rezultate predstavila predvsem v obliki grafov.

\hypertarget{glavna-ideja-programa-psevdokoda}{%
\subsection{Glavna ideja programa /
psevdokoda}\label{glavna-ideja-programa-psevdokoda}}

Problema sva se lotila tako, da sva sprva spisala program za generiranje
Voronoijevih celic iz grafa in šele nato kodo za generiranje želenih
grafov, v nadaljevanju pa bo program predstavljen v bolj logičnem
vrstnem redu.

\hypertarget{generiranje-grafov}{%
\subsubsection{Generiranje grafov}\label{generiranje-grafov}}

Končni cilj tega dela programa je vrniti matriko sosednosti za želen tip
grafa. Za vse grafe sva po predlogu prof. Vidalija skonstruirala razred
Graf, ter nato še poseben razred za vsak tip grafa posebej. Razredi
omogočajo generiranje različnih matrik sosednosti za vsak tip grafa, ter
poenostavijo genriranje Voronoijevih celic za vsak tip, saj omogočajo
enostavno klicanje za vsak tip, ter različna števila vozlišč grafov.\\
Poleg tega vsakemu izmed razredov pripada posamezna funkcija, s pomočjo
katere se skonstruira primerna matrika sosednosti za dan tip grafa. Te
funkcije sva skonstruirala za vsak tip grafa, glede na njihove
zakonitosti npr. v 2D mrežah ima vsako vozlišče lahko največ 4 sosednja
vozlišča, v binomskih drevesih pa največ 3. 3D mreže so skonstruirane na
podoben način kot 2D mreže, le v treh dimenzijah, cikli pa se generirajo
iz binomskih dreves z dodajanjem povezav. Natančnejša razlaga kako sva
konstruirala posamezen tip grafa je na voljo v komentarjih ob kodi v
programu samem.

\hypertarget{konstruiranje-voronoijevih-celic-za-posamezen-graf}{%
\subsubsection{Konstruiranje Voronoijevih celic za posamezen
graf}\label{konstruiranje-voronoijevih-celic-za-posamezen-graf}}

Ko enkrat imamo program, ki nam skonstruira matriko sosednosti za
poljuben tip grafa določen z dimenzijami ali številom vzolšič, lahko
zgeneriramo Voronoijev diagram za ta graf. Za to najprej potrebujemo
razdalje med vsemi vozlišči znotraj grafa. Za to sva uporabila znan
Floyd-Warshallov algoritem, ki iz matrike sosednosti vrne najkrajše poti
med vsemi pari vozlišč. Nato pa te vrednosti uporabi glavna funkcija
programa, Voronoi, ki kot podatke vzame matriko sosednosti želenega
grafa, množico U vseh središč Voronoijevih celic, ter tip grafa (da vemo
za kateri tip grafa iščemo diagram), ter vrne seznam seznamov. V vsakem
od notranjih seznamov je prvi element središče Voronoijeve celice,
sledijo pa vsa ustrezna vozlišča.\\
Funkcija deluje tako, da za graf s pomočjo Floyd-Warshalla določi
najkrajše razdalje vseh vozlišč grafa do izbranih središč celic. Nato po
stolpcih dobljene matrike primerja vse razdalje med seboj in poišče
najkrajšo, ki jo potem doda v ustrezen seznam (k pripadajočemu
središču). Če je neko vozlišče enako oddaljeno do več središč, ga doda
na seznam k vsem središčem. Funkcija nato zapiše končni seznam seznamov
v obliki \emph{data frame}-a.

\hypertarget{generiranje-rezultatov}{%
\subsubsection{Generiranje rezultatov}\label{generiranje-rezultatov}}

Na koncu sva sestavila funkcijo, ki nam generira Voronoijeve diagrame za
določen tip in število vozlišč, ter jih zapisuje v tekstovno datoteko.
Odločila sva se, da želiva preveriti diagrame za čim več različih
vrednosti \(k\), saj bi tako dobila najboljše primerjave za grafe.
Vendar se je hitro izkazalo, da bo program precej časovno zahteven, zato
sva se odločila, da si bova ogledala le vrednosti \(k\) med 5 in 60
odstotki vseh vozlišč grafa, saj sva tam pričakovala najbolj zanimivo
dogajanje. Za recimo grafe s 100 vozlišči sva torej vzela
\(k=5, 6, 7, ..., 60\).\\
Odločila sva se tudi, da bi bilo smiselno generiranje za vsak graf
večkrat ponoviti. Ker je bilo spet očitno, da bo za velike grafe to
lahko časnovno zelo dolgotrajno, sva se odločila, da bova za grafe z
največ 100 vozlišči generirala 50 vzorcev, za večje pa le 20. Za
primerjavo - za konstruiranje diagrama s 100 vozlišči je program
potreboval povprečno 22 sekund, za 200 vozlišč povprečno 36 sekund, za
500 vozlišč pa že približno 25 minut, iz česar lahko razberemo, da je
časovna zahtevnost programa zelo eksponentna.\\
Vse rezultate za en tip grafa s stalnim številom vozlišč sva shranila v
isto tekstovno datoteko, ter na koncu programa dodala še funkcijo main,
s katero sva lahko generirala diagrame za več različnih tipov in števil
vozčišč hkrati.

\hypertarget{statistiux10dno-modeliranje-v-programu-r}{%
\subsection{Statistično modeliranje v programu
r}\label{statistiux10dno-modeliranje-v-programu-r}}

\hypertarget{obdelava-podatkov-v-programu-r}{%
\subsubsection{Obdelava podatkov v programu
r}\label{obdelava-podatkov-v-programu-r}}

Podakte sva uvozila v program r in jih statistično obdelala. Najprej sva
spisala funkcijo, ki je vsaki celici priredila status velike, povprečne
ali male, glede na število elementov v njej. Najprej sva preračunala
povprečno velikost vseh celic, glede na število središč. Kot velike
celice sva nato nastavila tiste, ki so bile od povprečnih večje za
faktor 1.9, kot male pa tiste, ki so bile manjše za faktor 0.1. Nato sva
za vsak tip celice preračunala povprečno velikost in jih prikazala na
grafu, za vsako število vozlišč posebej. V tem sklopu sva naredila še
primerjavo povprečnih velikosti celic glede na število središč in
vozlišč.\\
(za tu spiši kar si ti svojega obdeloval)

Za konec sva iz rezultatov poskušala napovedati, kako bi izgledali
rezultati pri grafih z več vozlišči.

\hypertarget{rezultati}{%
\subsubsection{Rezultati}\label{rezultati}}

(rezultati za mreže in 3D mreže po zgledu spodaj)

\includegraphics{porocilo_projekta_files/figure-latex/binomski-1.pdf}

\begin{center}\includegraphics{porocilo_projekta_files/figure-latex/grafi_binomski-1} \end{center}

\begin{center}\includegraphics{porocilo_projekta_files/figure-latex/grafi_binomski-2} \end{center}

\begin{center}\includegraphics{porocilo_projekta_files/figure-latex/grafi_binomski-3} \end{center}

Kot je zelo lepo razvidno iz zgornjih grafov, se za vse 3 (in tudi vse
neprikazane grafe) pojavlja isti vzorec. Ko je število središč majhno,
so povprečne velikosti celice občutno večje kot ob velikih številih
središč. To je seveda logično, saj je jasno, da je povprečno število
celic sorazmerno padajoče z večanjem števila središč. Morda se na grafih
nekoliko zavajajoče zdi, da imajo grafi s 100 in 500 vozlišči ob majhnem
številu središč enako velikost celic, vendar je treba opomniti, da se
graf za 500 vozlišč začne z 20 središči, graf za 100 vozlišči pa s 5, ko
primerjamo pri enakem številu središč se vrednosti občutno razlikujejo -
pri grafu s 100 vozlišči je pri dvajsetih središčih povprečna velikost
povprečne celice nekaj manj kot 10, medtem ko je pri grafu z 500
vozlišči ta velikost 30. Na vseh grafih je očitno eksponentno padajoče
sorazmerje, kar lahko predvidevamo za vse grafe z več vozlišči.

Zelo podobne rezultate dobimo tudi pri analiziranju ciklov, le da so
bile najvišje povprečne velikosti celic nekaj višje (za približno 10).
Še vedno pa velikosti eksponentno padajo z višanjem števila središč.

\includegraphics{porocilo_projekta_files/figure-latex/cikli-1.pdf}

\begin{center}\includegraphics{porocilo_projekta_files/figure-latex/grafi_cikel-1} \end{center}

\begin{center}\includegraphics{porocilo_projekta_files/figure-latex/grafi_cikel-2} \end{center}

\begin{center}\includegraphics{porocilo_projekta_files/figure-latex/grafi_cikel-3} \end{center}

Ko primerjamo grafe za različna števila vozlišč, je to eksponentno
padanje zelo lepo vidno.

\end{document}
