\documentclass[a4paper, 12pt]{article}
\usepackage[slovene]{babel}
\usepackage[utf8]{inputenc}
\usepackage[T1]{fontenc}
\usepackage{lmodern}
\usepackage{graphicx}
\usepackage{amsmath}
\usepackage{amsfonts}
\usepackage{float}
\usepackage[document]{ragged2e}

\title{%
	Voronoijevi diagrami \\
	\large Uvodno poročilo za Finančni Praktikum}

\author{
	Anže Mramor\\
	\and
	Peter Tiselj}


\begin{document}
\setlength{\parindent}{0,5cm}

\maketitle


\section{Definicija Voronoijevega diagrama}
Definiramo jih na ravnini ali na splošnem metričnem prostoru. V danem grafu $G$ in podmnožici vozlišč $U$ ($U \subseteq G$) razdelimo vozlišča grafa $G$, glede na to, katero vozlišče iz $U$ jim je najbližje. Bolj natančno: za $\forall u \in U$, definiramo Voronoijevo celico, $(u, U)$, kot množico vseh vozlišč iz $G$, ki so bližje $u$ kot kateremu koli drugemu vozlišu $U \setminus \{u\}$.\\
Oziroma s formalno definicijo:\\
Naj bo $X$ metrični prostor z razdaljo $d$, ki je definirana kot razdalja med točko $x \in X$ in podmnožico $A$ : $d(x, A) = inf \{d(x, A) | a \in A \}$ . Naj bo še $K$ množica indeksov in $(P_k)_{k \in K}$ terica (tuple) nepraznih podmnožic v $X$. Potem definiramo Voronoijevo celico kot $R_k$ za $k \in K$ kot
$$ R_k = \{ x \in X | d(x,P_{k}) \leq d(x,P_{j}) ; \forall j \neq k \} $$
Voronoijev diagram je enostavno terica Voronoijevih celic $(R_k)_{k \in K}$

\section{Opis problema}
Pogledala si bova enostavna omrežja:
\begin{itemize}
\item mreže različnih velikosti ( $1 \times n, 2 \times n, n \times n, \dots $)
\item različne tipe grafov (3 dimenzionalne mreže, unije ciklov, binarna drevesa in podobno)
\end{itemize}
Večino specifičnih omrežij si bova izbrala naknadno, ko bova dobila prve rezultate na recimo zgornjih primerih in potem videla kaj bi bilo še zanimivo oziroma smiselno preveriti.\\
Za vsako omrežje si bova izbrala več smiselnih vrednosti $k$, ki nama bo predstavljalo število naključno izbranih vozlišč, za katere bova izračunala Voronoijeve celice. Nato bova statistično obdelovala velikosti dobljenih Voronoijevih celic. Pogledala bova na primer povprečno število vozlišč v minimalni, maksimalni in povprečni Voronoijevi celici za vsako omrežje in vse izbrane $k$.\\
Za vsak izbrani $k$ bi bilo smiselno program pognati večkrat, da bova lahko opazovala kako se pri različnem izboru začetnih vozlišč, spreminjajo zgornje vrednosti. Rezultate bova beležila in jih predstavila v obliki tabel in grafov.

\section{Postopek oziroma potek dela / psevdo opis algoritma}
Vhodni podatek: povezana mreža ali graf vozlišč - $G$
\begin{itemize}
\item Naključno generiranje smiselne vrednosti $k$ 
\item Za vsak $k$ naključno izbremo vozlišča v danem omrežju (generiranje množice $U$)
\item Za vsak $u \in U$ s pomočjo Dijkstrovega algoritma/Floyd-Warshallovega izračunamo najkrajše poti do vseh vozlišč $v \in G \setminus U$
\item Skonstruiramo drevo najkrajših poti - za korene dreves uporabimo elemente množice $U$, razdalja do lista drevesa pa je definirana, kot v definiciji diagrama
\item Če ima en list enako razdaljo do dveh korenov, ga dodamo na koncu obeh dreves najkrajših poti
\item Drevesa najkrajših poti bodo predstavljala Voronoijeve celice in na njih bova izvajala statistično ocenjevanje - prešteli šteilo vozlišč v vsakem drevesu in izračunali želene vrednosti
\end{itemize}
Sledilo bo statistično modeliranje rezultatov.

\end{document}
